\documentclass{article}
\usepackage{baked}
\usepackage[left=1.5in, right=2in, top=0.45in, bottom=0.45in]{geometry}
\usepackage{fontspec}
\setmainfont{Adobe Caslon Pro}
\usepackage{layout}
\usepackage[inline]{enumitem}
\usepackage[]{amsmath}
\usepackage{amssymb}
\usepackage{physics}
\usepackage{notoccite}
\usepackage{ragged2e}
\usepackage{graphicx}
\usepackage{eufrak}
\usepackage{pdfpages}
\usepackage{cite}
\usepackage{booktabs}
\usepackage{tikz}
\usepackage{caption}
\usepackage{hyperref}
\usepackage{amsmath}
\usepackage{bm}
\pagestyle{empty}
\begin{document}
{
  \fontsize{12pt}{18pt}\selectfont
  \addfontfeature{Numbers=OldStyle}
  {\scshape ece}471, GANs for Type\\
  Advisor: Chris Curro
}
\raggedright
\raggedbottom
\fontsize{12pt}{16pt}\selectfont 
\addfontfeature{Numbers=OldStyle}

\begin{description}

\item[Core topics] \hfill
  
  \begin{itemize}[leftmargin=-0.25cm]
  \item Generative adversarial networks
  \item Bézier polynomials and interpolation
  \item Fast rasterization techniques on massively parallel computer
    architectures (graphics processing units, etc.)
  \item Parametric design
  \end{itemize}

\item[References] \hfill
  
  \begin{itemize}[leftmargin=-0.25cm]
  \item {\itshape Deep Learning}, Ian Goodfellow and Yoshua Bengio and
    Aaron Courville, MIT Press, 2016
  \item {\itshape Bézier and B-Spline Techniques}, Hartmut Prautzsch,
    Wolfgang Boehm, and Marco Paluszny, Springer, 2002
  \item {\itshape The Metafont Book}, Donald Knuth, Addison-Wesley
    Longman, 1986
  \item {\itshape Lettering \& Type: Creating Letters and Designing
    Typefaces}, Bruce Willen, Nolen Strals Princeton Architectural
    Press, 2009
  \item {\itshape The Anatomy of Type}, Stephen Cole, Harper Collins,
    2012
  \end{itemize}

\item[Deliverables] \hfill
  
  \begin{itemize}[leftmargin=-0.25cm]
  \item Core program and associated experiments
  \item Paper or papers intended to be submitted at a conference such
    as {\scshape siggraph} (preferred), {\scshape cvpr}, or similar
  \item Typographic specimens for exhibition at {\scshape eoys}
  \item Interactive typeface design application for exhibition at
    {\scshape eoys}
  \end{itemize}
    
\item[Prerequisite] \hfill
  \begin{itemize}[leftmargin=-0.25cm]
    \item Either:
      \begin{itemize}
      \item Computational Graphs for Machine Learning ({\scshape ece}471) 
      \item Frequentist Machine Learning ({\scshape ece}414)
      \end{itemize}
  \end{itemize}
  
\end{description}

We will develop a class and attribute conditional generative
adversarial network capable of producing vector graphics. No
generative model with vector graphic output has previously appeared in
literature. The generative model will be able to produce closed
shapes, with counter spaces, defined by a variable number of cubic
Bézier curves. The motivating application for this research is
algorithmic typeface design.


Time permitting, we further intend to develop an interactive typeface
editing application which will embed a pre-existing typeface into the
high-level low-dimensional space defined by our generative model. By
encoding an existing typeface in this way we are able to perform
vector arithmetic in the embedding space to produce fundamental
changes in the Bézier space representation of the typeface glyphs.

\end{document}

